%%% Local Variables:
%%% mode: latex
%%% TeX-master: "<none>"
%%% End:

\title{Golomb-Rice coding}

\author{Vicente González Ruiz}

\maketitle

\section{Basics}  
\begin{itemize}
\item
  When the probabilities of the symbols follow an
  \href{https://en.wikipedia.org/wiki/Exponential_distribution}{exponential
  distribution}, the Golomg encoder~\cite{golomb1966run} has the same efficiency than the
  Huffman coding, but it is faster. In this case, the probabilities of
  the symbols shoud be

  \begin{equation}
    p(s) =
    2^{\displaystyle-\Big(\displaystyle m\big\lfloor\displaystyle\frac{s+m}{m}\big\rfloor\Big)}
    \tag{Eq:Golomb}
  \end{equation}

  where \(s=0,1,\cdots\) is the symbol and \(m=0,1,\cdots\) is the
  ``Golomb slope'' of the distribution.
\item
  If \(m=2^k\), it is possible to find a very efficient implementation
  and the encoder is also named Rice encoder~\cite{rice1971adaptive}. In this case

  \begin{equation}
    p(s) =
    2^{\displaystyle-\Big(2^k \displaystyle\big\lfloor\displaystyle\frac{s+2^k}{2^k}\big\rfloor\Big)}
    \tag{Eq:Rice}
    \label{eq:Rice}
  \end{equation}
\end{itemize}

\imgw{600}{graphics/Golomb_coding.png}

\begin{itemize}
\item
  Notice that for \(m=1\), we take the unary encoding.
\item
  The following distribution is for \(m=2\):
\end{itemize}

\section{Golomb Encoder}
\begin{enumerate}
\def\labelenumi{\arabic{enumi}.}
\tightlist
\item
  \(k\leftarrow \lceil\log_2(m)\rceil\).
\item
  \(r\leftarrow s~\mathrm{mod}~m\).
\item
  \(t\leftarrow 2^k-m\).
\item
  Output \(\lfloor s/m \rfloor\) using an unary code. /* Most
  significant bits */
\item
  If \(r<t\):
  \begin{enumerate}
  \def\labelenumii{\arabic{enumii}.}
  \tightlist
  \item
    Output the integer encoded in the \(k-1\) least significant bits of
    \(r\) using a binary code. /* Least significant bits */
  \end{enumerate}
\item
  Else:
  \begin{enumerate}
  \def\labelenumii{\arabic{enumii}.}
  \tightlist
  \item
    \(r\leftarrow r+t\).
  \item
    Output the integer encoded in the \(k\) least significant bits of
    \(r\) using a binary code.
  \end{enumerate}
\end{enumerate}

\subsection{Example ($m=7$ and $s=8$)}
\begin{enumerate}
\def\labelenumi{\arabic{enumi}.}
\tightlist
\item
  {[}1{]} \(k\leftarrow \lceil\log_2(8)\rceil=3\).
\item
  {[}2{]} \(r\leftarrow 8~\text{mod}~7 = 1\).
\item
  {[}3{]} \(t\leftarrow 2^3-7 = 8-7 = 1\).
\item
  {[}4{]} Output \(\leftarrow \lfloor 8/7\rfloor=1\) as an unary code (2
  bits). Therefore, output \(\leftarrow 10\).
\item
  {[}5{]} \(r=1\le t=1\).
\item
  {[}6.A{]} \(r\leftarrow 1+1=2\).
\item
  {[}6.B{]} Output \(r=2\) using a binary code of \(k=3\) bits.
  Therefore, \(c(8)=10010\).
\end{enumerate}

\section{Golomb Decoder}
\begin{enumerate}
\def\labelenumi{\arabic{enumi}.}
\tightlist
\item
  \(k\leftarrow\lceil\log_2(m)\rceil\).
\item
  \(t\leftarrow 2^k-m\).
\item
  Let \(s\leftarrow\) the number of consecutive ones in the input (we
  stop when we read a \(0\)).
\item
  Let \(x\leftarrow\) the next \(k-1\) bits in the input.
\item
  If \(x<t\):
  \begin{enumerate}
  \def\labelenumii{\arabic{enumii}.}
  \tightlist
  \item
    \(s\leftarrow s\times m+x\).
  \end{enumerate}
\item
  Else:
  \begin{enumerate}
  \def\labelenumii{\arabic{enumii}.}
  \tightlist
  \item
    \(x\leftarrow x\times 2~+\) next input bit.
  \item
    \(s\leftarrow s\times m+x-t\).
  \end{enumerate}
\end{enumerate}

\subsection{Example (decode 10010 for $m=7$)}
\begin{enumerate}
\def\labelenumi{\arabic{enumi}.}
\tightlist
\item
  {[}1{]} \(k\leftarrow 3\).
\item
  {[}2{]} \(t\leftarrow 2^k-m = 2^3-7=1\)).
\item
  {[}3{]} \(s\leftarrow 1\) because we found only one \(1\) in the
  input.
\item
  {[}4{]} \(x\leftarrow \text{input}_{k-1} = \text{input}_2 = 01\).
\item
  {[}5{]} \(x=1\nless t=1\).
\item
  {[}6.A{]}
  \(x\leftarrow x\times x\times 2+\text{next input bit} = x\times 1\times 2+0 = 2\).
\item
  {[}6.B{]} \(s\leftarrow s\times m+x-t = 1\times 7+2-1=8\).
\end{enumerate}

\subsection{Lab}
TO-DO

\section{Rice Encoder}
\begin{enumerate}
\def\labelenumi{\arabic{enumi}.}
\tightlist
\item
  \(m\leftarrow 2^k\).
\item
  Output \(\leftarrow\lfloor s/m\rfloor\) using an unary code
  (\(\lfloor s/m\rfloor+1\) bits).
\item
  Output \(\leftarrow\) the \(k\) least significant bits of \(s\).
\end{enumerate}

\subsection{Example ($k=1$ and $s=7$)}
\begin{enumerate}
\def\labelenumi{\arabic{enumi}.}
\tightlist
\item
  {[}1{]} \(m\leftarrow 2^k=2^1=2\).
\item
  {[}2{]} Output \(\leftarrow \lfloor s/m\rfloor=\lfloor 7/2\rfloor=3\)
  using an unary code of 4 bits. Therefore, output \(\leftarrow 1110\).
\item
  Output \(\leftarrow\) the \(k=1\) least significant bits of \(s=7\).
  So, output \(\leftarrow 1\). Total output \(c(7)=11101\).
\end{enumerate}

\section{Rice Decoder}
\begin{enumerate}
\def\labelenumi{\arabic{enumi}.}
\tightlist
\item
  Let \(s\) the number of consecutive ones in the input (we stop when we
  read a 0).
\item
  Let \(x\) the next \(k\) input bits.
\item
  \(s\leftarrow s\times 2^k+x\).
\end{enumerate}

\subsubsection{Example (decode 11101 for $k=1$)}
\begin{enumerate}
\def\labelenumi{\arabic{enumi}.}
\tightlist
\item
  {[}1{]} \(s\leftarrow 3\) because we found \(3\) consecutive ones in
  the input.
\item
  {[}2{]} \(x\leftarrow\) next input \(k=1\) input bits. Therefore
  \(x\leftarrow 1\).
\item
  {[}3{]} \(s\leftarrow s\times 2^k+x = 3\times 2^1+1 = 6+1 = 7\).
\end{enumerate}

\subsection{Lab}
TO-DO

\bibliography{text-compression}
